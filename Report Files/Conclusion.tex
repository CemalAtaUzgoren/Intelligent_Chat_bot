\section{Conclusion}
This study introduces an innovative form of an interactive chatbot that displays the cheapest train fares across the UK, displays train arrival times and assists in forming contingency plans. Based on the crucial demand for chatbots that can rely on a GUI to continually query accurate train timings and fares and repeatedly deploy numerous web scraping scripts to discover the cheapest ticket, this project established a model with clear advantages for the display of train arrival time. The user’s destination data is collected via a GUI and fed to the developed predictive models of XGBoost, Random Forest and K-Nearest Neighbours to derive findings. The predictive models exhibited robustness and even a margin of error of less than a few minutes, testifying to the reliability of the chip-processing logic of the chatbot. %The extraction of train times includes several web scraping methods which require ethical consideration to avoid the misuse of the scripts.\vspace{0.5cm}

\noindent
Furthermore, the chatbot incorporated extensive contingency plans capable of providing rapid and thorough responses to a variety of situations, including route obstructions. To complement this, a KB (knowledge base) containing information about various scenarios and regulatory papers enabled the chatbot to assist in training operators and establishing its credibility. The project's successful outcome was probably based on the combination of ML (machine learning) and NLP (natural language processing) technologies that can help understand and respond to varying types of user requests in a correct contextual manner. Extensive testing, including unit testing, integration testing, and usability testing, was carried out to prove that the chatbot performed well and that users were satisfied.\vspace{0.5cm}

\noindent
This project also emphasises the ethical aspects of online scraping and data protection, following best practices to prevent overloading servers and obtaining personal information. By avoiding commercial APIs and using ethical web scraping techniques, the project struck a compromise between usefulness and moral responsibility.\vspace{0.5cm}

\noindent
Finally, this study displays substantial advances in chatbot capabilities by combining natural language processing and machine learning to improve user interactions in the rail travel area. Future improvements include broadening the chatbot's knowledge base, enhancing its NLP capabilities for more sophisticated comprehension, and incorporating more transit options to create a more comprehensive travel assistant. The success of this initiative demonstrates the potential for intelligent chatbots to improve customer service and operational efficiency in the transportation industry.